\section{Plan eksperymentów}
W~ramach projektu wykonane zostaną trzy eksperymenty:
\begin{itemize}
    \item badanie ogólnego zachowania wskaźników jakości w~zależności od parametrów funkcji jądrowych,
    \item optymalizacja parametrów funkcji jądrowej w~celu uzyskania jak najlepszego modelu,
    \item badanie wpływu szumu na jakość modelu.
\end{itemize}

W~celu zbadania ogólnego zachowania wskaźników jakości w~zależności od parametrów funkcji jądrowych, wykorzystany zostanie algorytm przeszukiwania zupełnego po przestrzeni dostępnych parametrów. Badana przestrzeń zostanie podzielona w~taki sposób aby rozważać tylko punkty o~różnych rzędach wielkości. Dla każdego zestawu parametrów zostanie przygotowany model SVR, którego wskaźniki jakości zostaną obliczone za pomocą pięciokrotnej walidacji krzyżowej.

Do interpretacji uzyskanych wyników, zostaną wykorzystane reguły asocjacyjne. Takie podejście pozwoli na automatyczne wyciągnięcie wniosków na temat pożądanych wartości poszczególnych parametrów, w~przypadku gdy nie będzie możliwym wykreślenie charakterystyki na wykresie.

W~drugim kroku, za pomocą algorytmu ewolucyjnego dokonamy optymalizacji parametrów ze względu na badane wskaźniki jakości. Korzystając z~wiedzy z~pierwszego eksperymentu, zawęzimy przedział poszukiwanych parametrów, tak aby ograniczyć ilość potrzebnych obliczeń do minimum. Dla każdej badanej funkcji jądrowej, proces optymalizacji zostanie przeprowadzony ze względu na jeden z~badanych wskaźników jakości: błąd średniokwadratowy, średni błąd względny oraz współczynnik determinancji.

Tematem ostatniego eksperymentu, będzie zbadanie działania zoptymalizowanych modeli w~obliczu zaszumionych danych. Na oryginalną funkcję~\ref{fig:fun} zostanie nałożony biały, addytywny, gaussowski szum o~różnych poziomach mocy. Dla każdego badanego poziomu, zostaną zbadane wskaźniki jakości zarówno na zbiorze trenującym jak i~zbiorze walidacyjnym. Celem tego eksperymentu będzie określenie dla jakiego poziomu szumu model zaczyna znacząco tracić na swojej jakości.

Projekt zostanie zrealizowany głównie przy użyciu języka Python, wykorzystując środowisko Jupyter Notebook. Do generacji badanej funkcji oraz wszelkich obliczeń numerycznych zostanie użyty pakiet \textbf{numpy}~\cite{numpy}. Do tworzenia maszyn wektorów nośnych wykorzystany zostanie pakiet \textbf{scikit-learn}~\cite{scikit-learn}, który udostępnia moduł \textbf{svm}. W~module tym znajdują się obiekty \textbf{SVM} (do klasyfikacji) oraz \textbf{SVR} (do regresji). 

\begin{lstlisting}[language=Python, captionpos=b, caption=Nagłówek klasy sklearn.svm.SVR]
class sklearn.svm.SVR(*, kernel='rbf', 
      degree=3, gamma='scale', 
      coef0=0.0, tol=0.001, C=1.0, 
      epsilon=0.1, shrinking=True, 
      cache_size=200, verbose=False, 
      max_iter=-1)
\end{lstlisting}
