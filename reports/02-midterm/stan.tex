\subsubsection{Badanie wpływu parametrów funkcji jądrowych na wskaźniki jakości modelu}
Przeprowadziliśmy badania czterech funkcji jądrowych oraz wpływu ich parametrów na trzy różne wskaźniki jakości. W~tabeli~\ref{tab:modele} porównane zostały konkretne funkcje, badane parametry oraz liczbę policzonych modeli.

\begin{table}[tb]
 \centering
 \begin{tabular}{||c c c||} 
 \hline
 funkcja jądrowa & parametry & liczba modeli \\ [0.5ex]
 \hline\hline
 liniowa & $C$,$\varepsilon$ & 125  \\ 
 \hline
 rbf & $C$,$\varepsilon$,$\gamma$ & 625 \\
 \hline
 wielomianowa & $C$,$\varepsilon$, $d$, $c_{0}$ & 3125 \\
 \hline
 sigmoidalna & $C$,$\varepsilon$, $\gamma$, $c_{0}$ & 3125 \\
 \hline 
\end{tabular}

 \caption{Porównanie badanych funkcji jądrowych \label{tab:modele}}
\end{table}

Dla funkcji liniowej, która ma tylko dwa stopnie swobody, możliwym jest wykreślenie zależności danego wskaźnika jakości od parametrów $C$ oraz $\varepsilon$. Zależności te zostały wykreślone na rysunkach~\ref{fig:mse} oraz~\ref{fig:r2}.

\begin{figure}[h]
    \centering
    \includegraphics[width=0.5\textwidth]{assets/linear-mse.pdf}
    \caption{Wartość MSE w~zależności od $C$ oraz~$\varepsilon$}
    \label{fig:mse}
\end{figure}

\begin{figure}[h]
    \centering
    \includegraphics[width=0.5\textwidth]{assets/linear-r2.pdf}
    \caption{Wartość $R^2$ w~zależności od $C$~oraz~$\varepsilon$}
    \label{fig:r2}
\end{figure}

Dla wyników przeszukiwania odnoszących się do pozostałych funkcji jądrowych, wygenerowane zostały reguły asocjacyjne. Najciekawsze reguły zostały przedstawione w tabeli~\ref{tab:reguly}.

Kod realizujący przeszukiwanie zupełne po podanej przestrzeni parametrów został zamieszczony w~notatniku \texttt{Projekt-WMH.ipynb} natomiast analiza wyników została przeprowadzona w~notatniku \texttt{GridSearchResults.ipynb}. Indukcja reguł odbyła się przy użyciu języka~R, skrypt wykonujący to zadanie można znaleźć pod ścieżką \texttt{rules-induction/rules.R}.

\subsubsection{Optymalizacja modeli algorytmem ewolucyjnym}
Na podstawie wyników z~poprzedniej sekcji, ustaliliśmy sensowne przedziały dla każdego badanego parametru i~następnie uruchomiliśmy proces optymalizacji algorytmem ewolucyjnym. Na moment tworzenia tego dokumentu, udało nam się znaleźć optymalne parametry maszyny wektorów nośnych dla liniowej i~radialnej funkcji jądrowej. Wyniki działania procesu optymalizacji, wraz z~odpowiadającymi im wskaźnikami jakości, zostały przedstawione w~tabeli~\ref{tab:optim}. Podane wskaźniki jakości zostały obliczone \textbf{na zbiorze testowym}.

\begin{table}[h]
 \centering
 \begin{tabular}{||c c c c||} 
 \hline
 parametry & MAE & MSE & $R^2$ \\ [0.5ex]
 \hline\hline
 $C=64,76$, $\varepsilon=0,0067$ & $2,18\mathrm{e}{-3}$ & $7,173\mathrm{e}{-6}$ & $0,99$ \\ 
 \hline
 $C=21,62$, $\varepsilon=0,0134$ & $4,36\mathrm{e}{-3}$ & $2,856\mathrm{e}{-5}$ & $0,99$ \\ 
 \hline
 $C=6,092$, $\varepsilon=2,85\mathrm{e}{-3}$ & $9,13\mathrm{e}{-4}$ & $1,246\mathrm{e}{-6}$ & $0,99$ \\ 
 \hline

\end{tabular}

\caption{Wyniki procesu optymalizacji wskaźników jakości za pomocą algorytmu ewolucyjnego dla liniowej funkcji jądrowej \label{tab:optim}}
\end{table}

\begin{table*}[t]
 \centering
 \begin{tabular}{||c c c c||} 
 \hline
 poprzednik & następnik & wsparcie & ufność \\ [0.5ex]
 \hline\hline
 kernel=linear, C=small, epsilon=small & mean\_{}train\_{}mse=big & 0,133 & 0,666 \\ 
 \hline
 kernel=linear, C=small, epsilon=small & mean\_{}train\_{}mae=big & 0,133 & 0,666 \\ 
 \hline
 kernel=linear, C=small, epsilon=small & mean\_{}train\_{}r2=big & 0,133 & 0,666 \\ 
 \hline
 kernel=poly, C=big, coef0=small, degree=big, epsilon=small & mean\_{}test\_{}mae=small & 0,075 & 1,0 \\ 
 \hline
 kernel=poly, C=small, coef0=big, degree=small, epsilon=small & mean\_{}test\_{}mae=big & 0,05 & 1,0 \\ 
 \hline
 kernel=poly, C=big, coef0=big, degree=big, epsilon=small & mean\_{}fit\_{}time=big & 0,05 & 0,666 \\
 \hline
 kernel=rbf, C=big, epsilon=small, gamma=big & mean\_{}train\_{}mae=big & 0,144 & 1,0 \\
 \hline
 kernel=rbf, C=big, epsilon=small, gamma=big & mean\_{}test\_{}r2=big & 0,144 & 1,0 \\
 \hline
 kernel=rbf, C=big, epsilon=small, gamma=big & mean\_{}fit\_{}time=big & 0,144 & 1,0 \\
 \hline
 kernel=sigmoid, C=big, epsilon=small, coef0=small & mean\_{}train\_{}r2=big & 0,072 & 0,75 \\
 \hline
 kernel=sigmoid, C=small, epsilon=small, coef0=big & mean\_{}train\_{}mae=big & 0,064 & 0,666 \\
 \hline
 kernel=sigmoid, C=small, epsilon=small, gamma=small & mean\_{}fit\_{}time=big & 0,064 & 0,666 \\
 \hline
\end{tabular}

\caption{Najciekawsze odnalezione reguły asocjacyjne \label{tab:reguly}}
\end{table*}
