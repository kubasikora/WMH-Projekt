\subsection{Algorytm ewolucyjny}
Algorytm ewolucyjny to metoda optymalizacyjna przeszukująca przestrzeń rozwiązań, którego idea została zaczerpnięta z ewolucji. Niezależnie od rozwiązywanego problemu, pojęcia związane z tą metodą zostały użyczone bezpośrednio z biologii. Kolejne generacje gatunku mają być jak najlepiej przystosowane do otaczającego środowiska, eliminując zaadaptowane osobniki. 

\textit{Osobniki}, czyli podstawowe jednostki (przykładowe rozwiązania) podlegające ewolucji, której celem jest stworzenie reprezentanta (znalezienie rozwiązania) możliwie najlepszego. \textit{Fenotypem} nazywamy wygląd zewnętrzy osobnika (czyli funkcja końcowa), a \textit{genotypem} zbiór informacji, stanowiący jego pełen opis. \textit{Populacja} to z kolei zespół osobników przebywających we wspólnym
środowisku. Genotyp jest stały w trakcie życia osobnika, a modyfikacje następują w wyniku rozmnażania. Fenotyp odzwierciedla dopasowanie osobnika do środowiska i to na jego podstawie dokonywana jest selekcja. Na zmiany w fenotypie wpływają zmiany w genotypie, które są głównie efektem krzyżowania osobników, chociaż mogą też wynikać z mutacji-- losowych, niewielkich zmian genotypu.

Algorytm rozpoczyna wybranie losowo pewnej populacji. Na podstawie ich dopasowania do środowiska, dokonywana jest selekcja-- najlepszym osobnikom umożliwia się reprodukcję. Genotypy wybranych osobników poddawane są krzyżowaniu, a dodatkowo losowo wprowadzane są mutacje. W ten sposób powstaje kolejne pokolenie, potencjalnie doskonalsze. Utrzymanie stałej liczby osobników w populacji umożliwia usuwanie najsłabszych osobników, ocenianych na podstawie fenotypu (funkcji go oceniającej). Jeśli nie zostanie spełnione kryterium stopu, powraca się do procesu reprodukcji.

Modyfikacje algorytmu ewolucyjnego uwzględniają różne definicje operacji krzyżowania, mutacji oraz selekcji. To co charakteryzuje tego typu algorytmy to szybkiego, równoległego przeszukiwania przestrzeni oraz uniknięcie pułapek minimum lokalnego.
